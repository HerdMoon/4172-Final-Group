\documentclass[10pt]{article}
\usepackage{graphicx} 
\usepackage[usenames, dvipsnames]{color}
\usepackage{listings}
\usepackage{xcolor}
\lstset { %
    language=C++,
    backgroundcolor=\color{black!5}, % set backgroundcolor
    basicstyle=\footnotesize,% basic font setting
}
\usepackage[CJKbookmarks]{hyperref} 
\usepackage{color,soul}
\definecolor{lightblue}{rgb}{.90,.95,1}
\sethlcolor{lightblue}

%opening
\title{Overview and Permission}
\author{Team 8: Bihao Zhang(bz2324), Chihao Feng (cf2704), \\Chenqin Xu (cx2198), Siyuan Yu(sy2746)}

\begin{document}
\maketitle
\section{Overview}
Our project is an augmented-reality application, which would provide great convenience for discovering, creating and recording craft making in the lab. This application saves the trouble of opening a lot of drawers to find the target material by allowing users to get the information about materials in a drawer without actually opening it. Also it encourages users to take photos of their completed handcrafts so that other users can get to know how a product of a recipe would look like. There are 3 functions in the application in the main menu: \textcolor{blue}{select material}, \textcolor{blue}{save picture} and \textcolor{blue}{scan drawers}. New users can user \textcolor{blue}{scan drawers} to scan all the image target posted on drawers and then the application will know the position of all drawers. Users use the \textcolor{blue}{select material} to show materials information in selected drawers and find paths to related materials of their choosing recipes. The \textcolor{blue}{save picture} function allows users to take pictures of their completed handcraft work and selected related materials and recipe. They then upload this picture, which will show up in the information panel in "select material" function. Our application is easy to use and greatly improve the efficiency and enjoyment of handcraft making.


\section{Permission}
This is a permission statement for use of our media and names online of our course project by the professor and TAs.

\end{document}
